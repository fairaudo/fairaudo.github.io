abstract: "As a consequence of the policy responses to the COVID-19 crisis, central bank balance sheets, public debt and liquidity increased in many developed economies. As the economies recover and inflation far exceeds the target, central banks face a challenge in how to manage their balance sheet. I study the macroeconomic effects of reducing the central bank balance sheet size, i.e., Quantitative Tightening (QT). I construct a Regime-Switching New Keynesian DSGE model calibrated to the US economy before the COVID-19 crisis. The economy fluctuates between a monetary-led regime, a fiscally-led regime, and the zero lower bound on the monetary policy interest rate. The macroeconomic effects of QT crucially depend on the fiscal-monetary policy mix. In a monetary-led regime, QT reduces inflation at the cost of increasing the government debt-to-GDP ratio. The effects on output, inflation, and debt depend on the strategy’s aggressiveness. In contrast, unwinding the central bank balance sheet in a fiscally-led regime has little impact on inflation. The negative demand effect driven by QT is not enough to counteract the stimulative impact of negative real interest rates and fiscal stimulus."